\addchap{Заключение}

В ходе выполнения данной лабораторной работы были выполнены следующие задачи:
\begin{itemize}
	\item изучены основные понятия теории принятия решений;
	\item рассмотрены этапы процесса принятия решений;
	\item проведено ознакомление с имеющимися методами решения задачи многокритериального выбора;
	\item проведен сравнительный анализ рассмотренных методовх.
\end{itemize}

Для решения проблемы принятия выбора свойственно выделение альтернатив и их критериев, на основе которых производится дальнейшее рассмотрение. Также одним из важнейших этапов для многих методов выбора лучшей альтернативы является выделение множества Парето и его последующая аппроксимация. 
% множество Парето это кто, почему о нем не говорилось раньше
% не должно быть новой информации

Методы сравнения многокритериальных альтернатив могут не учитывать предпочтения ЛПР (методы глобального критерия и справедливого компромисса), получать информацию о них до формирования множества решений (методы скалярной свертки, $\varepsilon$-ограничений и целевого программирования), после выделения какой-то группы решений (методы уступок и Парето-силы) или же контролироваться ЛПР на определенных этапах до достижения желаемого результата (методы анализа иерархий, FFANN и IEM).

Для выбора конкретной группы методов необходимо предварительно определить, какая из целей является решающей --- простота вычислений, высокая точность или степень влияния ЛПР на ход решения.