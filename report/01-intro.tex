\setcounter{page}{4}
\addchap{Введение}

В современном мире необходимо выполнять множество действий. Из-за невозможности выполнения нескольких заданий одновременно приходится выбирать более приоритетные. Однако возникает вопрос --- каким образом расставлять приоритеты и от чего отталкиваться в случае, если это решение не поддается элементарной ранжировке?

Данную проблему можно обозначить как сравнение нескольких альтернатив \cite{Larychev}, которые можно охарактеризовать рядом критериев, по которым и будет осуществляться поиск оптимального решения. Все решается достаточно просто, если критерий всего один, но намного сложнее, когда их становится больше и они представляют собой разные категории оценки.

В таком случае необходимо воспользоваться методами, выделяющими лучшую комбинацию значений критериев из предоставленных. Определение наилучших значений может быть выполнено путем усреднения или же на основе личных предпочтений лица, осуществляющего процесс выбора.

Целью данной работы является изучение методов решения задачи многокритериальной оптимизации \cite{mko}. Для достижения поставленной цели необходимо выполнить следующие задачи:
\begin{itemize}
	\item изучить основные понятия теории принятия решений;
	\item рассмотреть этапы процесса принятия решений;
	\item ознакомиться с имеющимися методами решения задачи многокритериального выбора;
	\item провести сравнительный анализ рассмотренных методов.
\end{itemize}