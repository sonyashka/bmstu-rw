\chapter{Аналитический раздел}

В данном разделе будут представлены обзор предметной области, существующих методов  решения задачи многокритериальной оптимизации, а также проведен сравнительный анализ этих методов.

\section{Обзор предметной области}

\subsection{Основные понятия}

Проблема выбора является неотъемлемой частью нашей жизни. Она разрешается посредством <<принятия решения>>. Под этим понятием подразумевают выбор среди нескольких предложенных вариантов действия (далее --- альтернатив \cite{Larychev}), относящихся к какому-либо событию. Если не представлена ни одна из альтернатив, то проблема выбора отсутствует.

Альтернативы бывают независимыми и зависимыми. Независимыми можно считать те альтернативы, при выполнении действий над которыми (удаление из рассмотрения, выделение как единственной лучшей), качество других альтернатив остается неизменным. В случае зависимых оценки одних из них оказывают влияние на качество других. В свою очередь зависимые альтернативы могут иметь различные типы зависимости. Например, непосредственная групповая зависимость --- в случае рассмотрения одной из групповых альтернатив, стоит обратить внимание и на остальные, принадлежащие этой группе.

За выбор какой-либо альтернативы (выделения ее из числа остальных) является ответственным лицо, принимающее решения (ЛПР) \cite{Larychev}. Наряду с ним можно также выделить владельца проблемы, активную группу и эксперта. В \cite{Larychev}
рассматривается применимость в сфере электронно-вычислительных машин (ЭВМ), и ввиду этого указывается, что все роли можно свести к ЛПР. Далее будут рассмотрены математическое и программное обеспечение, имеющее назначением поддержку принятия решений.

Выбор всегда осуществляется на основе имеющейся информации по каждой из альтернатив. В качестве характеристики их привлекательности для ЛПР используется термин <<критерии>> \cite{Larychev}. Для большинства задач можно выделить немалое множество критериев, которые также могут быть независимыми и зависимыми. Зависимыми называются те критерии, при которых оценка альтернативы по одному из них определяет оценку по другому критерию. Зависимость критериев между собой приводит к целостному выделению групп альтернатив. Количество выделенных критериев также сказывается на сложность проблемы принятия решения.

\subsection{Процесс принятия решения}

В процессе принятия решения можно выделить 3 основных этапа:
\begin{itemize}
	\item поиск информации;
	\item нахождение альтернатив и выделение их критериев;
	\item непосредственный выбор лучшей альтернативы.
\end{itemize}

На первом этапе происходит сбор всей информации, доступной на данном этапе. Второй этап включает в себя определение возможных действий в данной ситуации, формирование из них альтернатив и выявление факторов, способных повлиять на привлекательность или непривлекательность для каждой альтернативы. Данные действия являются трудоемкими и требуют тщательного подхода.

% !! не 1 этап, а и критерии, и шкалы оценок м.б. по приведенным ниже группам шкал
% точнее, сами критерии м.б.разными
% см. статью Ю.В. Строганов, Л.Л. Волкова -- nps.itas.miem.edu.ru - 2015 год, сборник статей - ссылка на сборник есть на 2й странице полного pdf, в конец вместо количества страниц дописываете -- С. ххх--ууу.
%Если Вы рассматриваете вкратце два первых этапа, то Вы можете вкратце сказать об обоих и тогда уже говорить, что в фокусе данной НИР находится последний этап. Либо нужно дать какую-то отсылку к источникам, где описаны первые 2, и всё равно заявить о фокусе

В ходе последнего этапа, являющегося фокусом данной работы, осуществляется оценка альтернатив по критериям. При этом выделяют следующие типы критериев \cite{param_classes}:
\begin{itemize}
	\item номинальные (качественные) --- определяется рядом состояний, каждый из критериев может находится только в одном из состояний;
	\item порядковые --- определяется упорядоченным набором состояний (порядок важен);
	\item численные (количественные) --- представляют собой измеримые или исчислимые количества, в качестве частных случаев можно выделить дискретные и непрерывные.
\end{itemize}

Для оценки конкретных критериев используют шкалы оценок \cite{Larychev}. Имеются следующие группы:
\begin{itemize}
	\item шкала порядка --- упорядочивание оценок по возрастанию/убыванию качества критерия;
	\item шкала равных интервалов --- для данной шкалы имеются равные расстояния по изменению качества между оценками, выбор начала и шага отсчета является произвольным;
	\item шкала пропорциональных оценок --- данный тип схож с предыдущим, но имеет особенность --- начало отсчета является не произвольным, а с экспериментально установленного нулевого пункта.
\end{itemize}

% написать о множестве Парето, описать различия П и Э-П множеств
Также в ходе принятия решения предварительной задачей является выделение множества Парето (или Эджворта-Парето) \cite{pareto}. 

Пусть существуют два альтернативных решения: А и Б; А не хуже, чем Б; оценка А по одному критерию лучше Б, тогда А называется доминирующим решением, а Б доминируемым. Альтернативы, не находящиеся в отношении доминирования, называются несравнимыми \cite{Larychev}.

Множество Парето содержит в себе множество всех несравнимых альтернатив. Данному множеству соответствует Парето-фронт \cite{pareto_f} --- образ Парето-множества в пространстве целевых функций. В свою очередь целевая функция является количественной мерой оптимальности решения.

Если имеет место задача поиска лучшей альтернативы, ее решение всегда оказывается внутри множества Парето. Поэтому для многих методов выделение данного множества является немаловажным этапом.

\newpage
\section{Имеющиеся решения задачи выбора среди многокритериальных альтернатив}

Существующие методы можно разделить на несколько групп:
\begin{itemize}
	\item не учитывающие предпочтения ЛПР --- задача состоит в поиске некоторого компромиссного решения;
	\item апостериорные --- предполагает внесение ЛПР информации о его предпочтениях после получения некоторого множества недоминируемых решений;
	\item априорные --- в отличие от предыдущей группы ЛПР вносит информацию о предпочтениях до начала решения задачи;
	\item интерактивные (адаптивные) --- методы данной группы состоят из совокупности итераций, каждая из которых включает в себя этап анализа, выполняемый ЛПР, и этап расчета, осуществляемый системой.
\end{itemize}

В таблицах \ref{mko_table_0} - \ref{mko_table_5} представлены методы рассмотренных групп и их описание.

\begin{table}[H]
	\centering
	\caption{Методы сравнения многокритериальных альтернатив}
	\label{mko_table_0}
	\begin{tabular}{|p{3.3cm}|p{2.4cm}|p{9.5cm}|}
		\hline
		\textbf{Группа\linebreak методов} & \textbf{Метод} & \textbf{Описание метода} \\
		\hline
		\multirow{2}{3.3cm}{Не учитывающие\linebreakпредпочтения ЛПР} & Метод глобального критерия & Суть данного метода состоит в том, что сначала определяется главный критерий, все остальные критерии формируются в ограничения выбранной целевой функции: каждому <<не главному>> критерию задается
		ограничение. Само ограничение должно удовлетворять требованиям, поставленным
		ЛПР, устанавливаются показатели или нижняя граница критериев. В итоге, задача сводится к решению задачи с одним критерием.\\
		\cline{2-3} & Метод справедливого компромисса & Справедливым \cite{compromise} считается такой компромисс, при котором относительный уровень снижения качества по одному или нескольким частным критериям не превосходит относительного уровня повышения качества по остальным частным критериям (меньше или равен). Для формализации данного метода вводится понятие превосходства альтернатив \cite{compromise}. Поиск такой альтернативы ведется до тех пор, пока она не станет единственной превосходящей или не достигнуто определенное число итераций.\\
		\hline
		Апостериорные & Метод\linebreakуступок & Поиск и принятие решения при использовании данного метода осуществляется по следующему алгоритму.
		\begin{enumerate}
			\item Предварительное ранжирование всех локальных критериев по важности; 
			\item Определение самого важного критерия и поиск наиболее оптимального решения по нему;
			\item Выбор следующего критерия по важности. Как и в п. 2 определяется
			оптимальное решение, но с тем отличием, что допускается потеря важности
			предыдущего критерия на какую-то величину (уступку);
		\end{enumerate}\\
		\hline
	\end{tabular}
\end{table}

\begin{table}[H]
	\centering
	\caption{Методы сравнения многокритериальных альтернатив (продолжение)}
	\label{mko_table_1}
	\begin{tabular}{|p{3.3cm}|p{2.4cm}|p{9.5cm}|}
		\hline
		\textbf{Группа\linebreak методов} & \textbf{Метод} & \textbf{Описание метода} \\
		\hline
		\multirow{2}{3.3cm}{Апостериорные} & Метод\linebreakуступок & Процесс оптимизации решения по каждому критерию будет идти до тех пор, пока последний по важности критерий не будет рассмотрен.\\
		\cline{2-3} & Метод Парето-силы & Отличием от предыдущего метода является этап ранжирования, в ходе которого осуществляется аппроксимация множества Парето по силе \cite{pareto_f} каждого из решений --- количеству доминируемых альтернатив. Данное выделение альтернатив обладает рядом недостатков, которые решаются использованием альтернативного понятия слабости --- количество доминирующих альтернатив. Точнее говоря используют вариант слабости, называемый хилостью --- суммарная сила альтернатив, доминирующих над конкретной. Тогда целевая функция для конкретной альтернативы имеет вид:
		\begin{equation}
			f(X_{i}) = \frac{1}{1 + w_{i}},
		\end{equation}
		где $w_{i}$ --- хилость альтернативы.\\
		\hline
		\multirow{2}{3.3cm}{Априорные} & Метод скалярной свертки & Данный метод заключается в том, что при комбинировании частных критериев,
		получается один скалярный критерий, и все сводится к решению однокритериальной задачи.\linebreakДля начала необходимо определить весовые коэффициенты для каждого критерия, такие что сумма всех критериев равнялась 1. Если критерий $f_{i}$ имеет приоритет над $f_{j}$, то веса соответствуют правилу $\alpha_{i} \geq \alpha_{j}$. Необходимо построить новую целевую функцию \cite{upr}.\\
		\hline
	\end{tabular}
\end{table}

\begin{table}[H]
	\centering
	\caption{Методы сравнения многокритериальных альтернатив (продолжение 2)}
	\label{mko_table_2}
	\begin{tabular}{|p{3.3cm}|p{2.4cm}|p{9.5cm}|}
		\hline
		\textbf{Группа\linebreak методов} & \textbf{Метод} & \textbf{Описание метода} \\
		\hline
		\multirow{2}{3.3cm}{Априорные} & Метод скалярной свертки & Есть несколько видов свертки: мультипликативный и аддитивный. Для аддитивного метода целевая функция имеет вид:
		\begin{equation}
		f(X) = \sum\limits_{k=1}^{K} \alpha_{k} f_{k}(X),
		\end{equation}\linebreakВ случае мультипликативного метода:
		\begin{equation}
		f(X) = \prod_{k=1}^{K} f_{k}^{\alpha_{k}}(X),
		\end{equation}\\
		\cline{2-3} & Метод\linebreak$\varepsilon$-ограни-\linebreakчений & В данном методе в качестве скалярного критерия оптимальности используется самый важный из частных критериев, а остальные частные критерии учитываются с помощью ограничений. Дополнительной является информация о номере самого важного из частных критериев и о максимально допустимых значениях критериев.\\
		\cline{2-3} & Метод целевого программирования & Основан на ранжировании критериев по важности для ЛПР. Основная задача поиска решений включает в себя несколько последовательных подзадач по оптимизации каждого из критериев. При этом такая оптимизация осуществляется согласно целевой функции, и улучшение значения по одному критерию не может достигаться за счет ухудшения значения по более важному критерию. Таким образом, итоговым результатом будет обнаружение наилучшего решения поставленной задачи. \\
		\hline
	\end{tabular}
\end{table}

\begin{table}[H]
	\centering
	\caption{Методы сравнения многокритериальных альтернатив (продолжение 3)}
	\label{mko_table_3}
	\begin{tabular}{|p{3.3cm}|p{2.4cm}|p{9.5cm}|}
		\hline
		\textbf{Группа\linebreak методов} & \textbf{Метод} & \textbf{Описание метода} \\
		\hline
		\multirow{2}{3.3cm}{Интерактивные} & Метод\linebreakанализа иерархий & Метод анализа иерархий - математический инструмент системного подхода к задачам принятия решений, при помощи которого проблема структурируется в виде иерархии \cite{mko_methods}. Анализ и принятие решения осуществляется в несколько этапов.\begin{enumerate}
			\item Построение иерархической структуры, которая в самом
			простом и базовом варианте содержит три элемента: цель,
			критерии и альтернативы;
			\item Определение приоритетов, имеющих относительную важность, путем осуществления процедуры парных сравнений;
			\item Вычисление альтернативного решения с
			максимальным значением приоритета. Данные вычисления
			проводятся на основе синтеза приоритетов составленной
			ранее иерархической структуры.
		\end{enumerate}\\
		\cline{2-3} & Метод FFANN (с использованием нейронной сети прямого распространения) & В методе ЛПР оценивает предоставленные ему решения, задавая конкретные значения своей функции предпочтений по каждому отдельному решению. Входами нейронной сети являются компоненты нормализованного вектора частных критериев оптимальности, выходом --- значение функции предпочтений.\\
		\hline
	\end{tabular}
\end{table}

\begin{table}[H]
	\centering
	\caption{Методы сравнения многокритериальных альтернатив (продолжение 4)}
	\label{mko_table_4}
	\begin{tabular}{|p{3.3cm}|p{2.4cm}|p{9.5cm}|}
		\hline
		\textbf{Группа\linebreak методов} & \textbf{Метод} & \textbf{Описание метода} \\
		\hline
		\multirow{2}{3.3cm}{Интерактивные} & Метод FFANN (с использованием нейронной сети прямого распространения) & Можно выделить следующие этапы.
		\begin{enumerate}
			\item Генерация недоминируемых векторов на множестве допустимых значений, вычисление соответствующего векторного критерия оптимальности;
			\item Предоставление значений ЛПР. Если его удовлетворяет текущее лучшее решение --- завершение вычислений;
			\item Нормализация компонентов критериальных векторов;
			\item Назначение ЛПР каждому решению значения своей функции предпочтений;
			\item Обучение нейронной сети и вычисление значения целевой функции. Если значение является уникальным, то переход к п. 1, иначе --- переход к п. 2.
		\end{enumerate}\\
		\cline{2-3} & Метод IEM (Интерактивный эволюционный метод) & В данном методе предполагается, что ЛПР вносит свои предпочтения в систему в виде попарного сравнения отдельных альтернатив \cite{pareto}. В основе данного метода лежит скалярная свертка частных критериев оптимальности вида:
		\begin{equation}
			\psi(X) = -\sum_{k=1}^{K}(\lambda_{k}|\phi_{k}^* - \phi_{k}(X)|)^t,
		\end{equation}
		где $t$ --- параметр, который определяет функцию предпочтений; $\lambda_{k}$ --- компонент вектора весовых множителей; $\phi_{k}^*$ --- компонент идеального вектора частных критериев оптимальности.\\
		\hline
	\end{tabular}
\end{table}

\begin{table}[H]
	\centering
	\caption{Методы сравнения многокритериальных альтернатив (продолжение 5)}
	\label{mko_table_5}
	\begin{tabular}{|p{3.3cm}|p{2.4cm}|p{9.5cm}|}
		\hline
		\textbf{Группа\linebreak методов} & \textbf{Метод} & \textbf{Описание метода} \\
		\hline
		\multirow{1}{3.3cm}{Интерактивные} & Метод IEM (Интерактивный эволюционный метод) & Выделяются следующие этапы.
		\begin{enumerate}
			\item Генерация начальной популяции решений;
			\item Формирование множества решений путем приближенного построения множества Парето;
			\item Выбор из множества решений некоторого количества для оценки ЛПР;
			\item Построение функции аппроксимации предпочтений на основе полученной информации;
			\item Формирование нового множества решений путем приближенного построения множества Парето, но уже с использованием <<функции приспособленности>> отдельных решений;
			\item При достижении условия останова --- завершение вычислений, иначе --- переход на п. 2.
		\end{enumerate}\\
		\hline
	\end{tabular}
\end{table}

%!! табличку, пожалуйста, на одном листе и весь текст, который есть тут, идёт в табличку
В таблице \ref{pos_neg} приведен сравнительный анализ групп методов сравнения многокритериальных альтернатив.

\newpage
\begin{table}[H]
	\centering
	\caption{Анализ методов сравнения многокритериальных альтернатив}
	\label{pos_neg}
	\begin{tabular}{|p{3.3cm}|p{5.9cm}|p{6.0cm}|}
		\hline
		\textbf{Группа\linebreak методов} & \textbf{Достоинства} & \textbf{Недостатки}\\
		\hline
		Не учитывающие предпочтения ЛПР & Простота с вычислительной точки зрения. & Не гарантируют выбор лучшей альтернативы.\\
		\hline
		Апостериорные & Предоставляют высокую точность. & Равномерная аппроксимация требует больших вычислительных затрат. Более того с повышением точности аппроксимации задача выбора становится более затруднительной для ЛПР.\\
		\hline
		Априорные & В отличие от апостериорных, пропадает необходимость построения всего множества достижения в связи с предоставлением информации о предпочтениях ЛПР до решения задачи. & Редко используются в связи с тем, что зачастую ЛПР затруднительно сформулировать предпочтения до решения задачи.\\
		\hline
		Интерактивные & ЛПР может влиять на ход поиска оптимального решения. & Значительное замедление процесса из-за неопределенного количества итераций.\\
		\hline
	\end{tabular}
\end{table}